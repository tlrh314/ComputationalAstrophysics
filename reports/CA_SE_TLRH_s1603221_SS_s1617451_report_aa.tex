%                                                                 aa.dem
% AA vers. 8.2, LaTeX class for Astronomy & Astrophysics
% demonstration file
%                                                       (c) EDP Sciences
%-----------------------------------------------------------------------
%
%\documentclass[referee]{aa} % for a referee version
%\documentclass[onecolumn]{aa} % for a paper on 1 column
%\documentclass[longauth]{aa} % for the long lists of affiliations
%\documentclass[rnote]{aa} % for the research notes
%\documentclass[letter]{aa} % for the letters
%\documentclass[bibyear]{aa} % if the references are not structured
% according to the author-year natbib style

%
\documentclass{aa}

%
\usepackage{graphicx}
%%%%%%%%%%%%%%%%%%%%%%%%%%%%%%%%%%%%%%%%
\usepackage{txfonts}
%%%%%%%%%%%%%%%%%%%%%%%%%%%%%%%%%%%%%%%%
\usepackage{hyperref}
\hypersetup{pdfborder=0 0 0, colorlinks=true, linkcolor=black, urlcolor=blue,
citecolor=black}
% To add links in your PDF file, use the package "hyperref"
% with options according to your LaTeX or PDFLaTeX drivers.
%
\renewcommand{\thefootnote}{\fnsymbol{footnote}}
\newcommand{\Sun}[0]{\ensuremath{_{\odot}}}
\newcommand{\mytilde}{\raise.17ex\hbox{$\scriptstyle\mathtt{\sim}$}}
\begin{document}


   \title{Simulating Stellar Evolution with AMUSE}

   \subtitle{Computational Astrophysics (CA) assignment two}

   \author{T. Halbesma (1603221)
          \inst{1}
          \and
          S. Sultan (1617451)\inst{2}
          }

   \institute{Anton Pannekoek Instituut (API), University of Amsterdam,
              Science Park 904, 1098 XH Amsterdam
              \email{timo.halbesma@student.uva.nl}
         \and
             Informatics Institute, Section Computational Science, University of Amsterdam,
             Science Park 904, 1098 XH Amsterdam
             \email{shabaz.sultan@student.uva.nl}
             }

   %\date{Received September 15, 1996; accepted March 16, 1997}

% \abstract{}{}{}{}{}
% 5 {} token are mandatory

  \abstract
  % context heading (optional)
  % {} leave it empty if necessary
   {}
  % aims heading (mandatory)
   {To compare Henyey type and parameterized stellar evolution codes. To determine cluster age by fitting simulated Hertzsprung Russel diagrams for different cluster ages to observed color-magnitude diagrams. In particular for the Hyades and Pleiades clusters. To set up a model for gravitational dynamics of a stellar clusters combined with stellar evolution to take stellar evolution into account of gravitational systems.}
  % methods heading (mandatory)
   {We use the AMUSE framework to simulate HR diagrams for a given Salpeter initial mass function of a population of stars assumed to be generated at the same point in time. We neglect dust and gas in the cluster and assume a single metallicity for the cluster as a whole. Furthermore, we set up a bridge between a Barnes-Hut N-Body gravitational dynamics code and the SSE stellar evolution code.}
  % results heading (mandatory)
   {We find that the simulated Hyades cluster age is 680 Myr as compared the literature value of 625 $\pm$ 50 Myr. A simulated cluster over time has an increasing number of remnant stars and a constant number of stars in a giant phase.}
  % conclusions heading (optional), leave it empty if necessary
   {By simulating a cluster and fitting it to observed data of a real cluster initial conditions for that cluster can be derived. Setting up a simulation with said initial conditions allows study of behaviour of stars over a cluster's lifetime. By using results from stellar evolution code better gravitational simulations can be realized that take into account the changing masses of stars.}

   \keywords{Stellar Evolution --
                AMUSE --
                MESA -- EVTwin -- SSE -- SeBa
                -- Cluster age determination
               }

   \maketitle
%
%________________________________________________________________

\section{Introduction}
Once upon a time there was a star. The star was living quietly and peacefully whilst maturing. The evolution of this star has, and will have a profound effect on the life on Earth as this star is our Sun. The inhabitants of planet Earth were quite eager to discover how the star will evolve and how much more trouble- and worry free days our civilization would be left on our beloved habitat, our paradise Earth. A major reason to study stellar evolution is self preservation. During the last century much progress in this field has been made in understanding the equations governing the stellar structure and evolution. Additionally, during the last decade, the computational power has increased drastically allowing for detailed calculations of, for instance, the Solar structure and (future) evolution using the previously acquired theoretical knowledge. With the current computing power at hand, one no longer has to assume stars to be static. For instance, we are now able to numerically calculate gravitational systems over the nuclear timescales of stars, also accounting for the stellar evolution that is expected to happen in astrophysical systems on this timescale.

In this report, the authors have used the Astrophysical Multipurpose Software Environment, in short AMUSE \citep{2009NewA...14..369P, 2013CoPhC.183..456P, 2013A&A...557A..84P}. As part of AMUSE several community codes are readily available by just changing a single code line. This allows for swift comparison between stellar evolution codes. Moreover, AMUSE combines gravitational dynamics code, stellar evolution methods and hydrodynamical algorithms. One of the great strengths of AMUSE is this combination enabling to set up a bridge between gravitational dynamics and stellar evolution routines. At first, in Sec.~\ref{sec:algorithms} we compare four stellar evolution algorithms. Next, in Sec.~\ref{sec:isochrones}, we obtain theoretical Hertzsprung–Russell diagrams for a given cluster age, so-called isochrones. We fit the isochrones to observed color-magnitude diagrams of clusters Hyades and Pleiades. We continue by examining the mass loss rate due to stellar evolution in Sec.~\ref{sec:massloss}. Finally, we set up the bridge between the stellar evolution and gravitational dynamics code to simulate clusters more realistically in Sec.~\ref{sec:SE_GD}, albeit with the significant simplifications casu quo assumptions discussed in Sec.~\ref{sec:discussion}. Finally, we present the conclusions in Sec.~\ref{sec:conclusions}

\section{Code Comparison}\label{sec:algorithms}
On the one hand we use two Henyey-type codes: \textbf{MESA} \citep{2011ApJS..192....3P, 2013ApJS..208....4P} and \textbf{EVTwin} \citep{1971MNRAS.151..351E, 1972MNRAS.156..361E, 1973MNRAS.163..279E, 1973A&A....23..325E, 1994MNRAS.270..121H, 1995MNRAS.274..964P, 2001ASPC..229..157E, 2001ApJ...552..664N, 2002ApJ...575..461E, 2007A&A...464L..57S, 2004MNRAS.348..201E, 2008A&A...488.1007G}. On the other hand, we use two parameterized codes:  \textbf{SSE} \citep{2000MNRAS.315..543H} and \textbf{SeBa} \citep{1996A&A...309..179P, 2012A&A...546A..70T}. For a detailed explanation of the Henyey-type code we refer the reader to \cite{1959ApJ...129..628H}. As for parameterized code, we cite \cite{AMUSEdocumentation}: ``stars are parameterized by mass, radius, luminosity, core mass, etc. as functions of time and initial mass''. The star is evolved from zero-age main sequence onwards to obtain the model parameters. The algorithms are analytic fit-based which makes SSE and SeBa rapid. Table~\ref{tab:algorithms} shows a comparison of general properties for the different algorithms.

In our analysis, we will evolve a star of solar mass and solar metallicity up to the current age of the Sun of 4.57 $\pm$ 0.11 Gyr \citep{2002A&A...390.1115B}. By default, the stellar evolution codes in AMUSE assume that the solar metallicity $z = 0.02$.
In addition, we will run a simulation of a star with 10 M\Sun up to 10 Myr because we expect heavier stars to evolve more rapidly. Specifically, the expected main sequence (hydrogen burning) lifetime $\tau_H$ as a function of mass is given by \citep[][p. 347]{2012sse..book.....K}. Normalised to the Solar lifetime of 10 Gyr \citep{1993ApJ...418..457S} yields
\begin{eqnarray}
     \tau_H(M) = 10^{10} \cdot \left(\frac{M}{M\Sun}\right)^{-2.5},
\end{eqnarray} where $M$ is the mass of the star. For a 10 solar mass star we expect the star to stay on the main sequence when it is evolved until 30 Myr. We therefore choose 10 Myr to stay well under the post-main sequence evolution.

Table~\ref{tab:parameterspace} shows the input parameters and the results for the for different algorithms. Three interesting observations are made. Firstly, $L$, $M$ and $R$ are of order unity in solar units, as expected for solar evolution up to the current age of the Sun, save for the final stellar radius calculated using the EVTwin algorithm. This is attributed to a potentially failed build of the EVTwin algorithm which causes doubt whether to use EVTwin algorithm. In addition, for the 10 M\Sun star the final stellar radii are comparable for MESA, SSE and SeBa but EVTwin differs significantly. Secondly, the Henyey model calculation times are significantly larger than the parameterized algorithms. In this case, the Henyey runtime is an order of magnitude larger than the parameterized runtime. Lastly, SSE and SeBa yield the same results. This is not surprisingly as the algorithms are not expected to differ for one star because SeBa is a more elaborate algorithm consisting of both SSE and binary evolution code.

In order to find the same evolutionary state of an evolved star one can use a built in method in AMUSE that keeps track of the evolutionary state of a star. The input parameter space therefore need not be changed. It is interesting to perform an in-depth analysis of the final (output) parameters as a function of algorithm. We have not included this analysis in the report because we consider this outside the scope of this project because we continue the analysis using only the SSE algorithm.

\begin{table*}
    \caption{General properties of the Stellar Evolution codes \citep{AMUSEdocumentation}}
    \label{tab:algorithms}
    \centering
    \begin{tabular}{l l l l l l l l }
        \hline\hline
        Algorithm & Type & Binary/Single & Metallicity $z$ & Mass $m$ (M\Sun) \\
        \hline
        MESA & Henyey & Both & All, even 0 & 0.1 - 100 \\
        EVTwin & Henyey & Single\footnotemark[1] & 0.02 only & ? \\
        SSE & Parameterized & Single & 0.0001 - 0.03 & 0.1 - 100 \\
        SeBa & Parameterized & Both & 0.0001 - 0.03 & 0.1 - 100 \\
        \hline
    \end{tabular}
\end{table*}
\addtocounter{footnote}{1}
\footnotetext[\value{footnote}]{Binaries are not yet available in AMUSE EVTwin Interface.}

\setlength\tabcolsep{2pt}
\begin{table}
    \caption[]{Parameters and results of simulations. Here, $z$ is the metallicity, $t_{\rm end}$ the end time of the simulation, $L$ the luminosity, $M$ the mass, $R$ the radius and Runtime is the wall-clock runtime of the algorithm.}
    \label{tab:parameterspace}
    \begin{tabular}{lcccccc}
        \hline
        \noalign{\smallskip}
        Algorithm & $z$ & $t_{\rm end} (Myr)$ & $L$ (L\Sun) & $M$ (M\Sun) & $R$ (R\Sun) & Runtime (s) \\
        \hline
        \noalign{\smallskip}
        MESA & 0.02 & 4600 & 1.034 & 1 & 1.016 & 9.54 \\
        EVTwin  & 0.02 & 4600 & 1.027 & 1 & 0.1834 & 9.58 \\
        SSE & 0.02 & 4600 & 0.9585 & 1 & 0.9856 & 0.166 \\
        SeBa & 0.02 & 4600 & 0.9585 & 1 & 0.9856 & 0.554 \\
        MESA & 0.02 & 10 & 7460 & 10 & 5.091 & 10.9 \\
        EVTwin & 0.02 & 10 & 7637 & 10 & 0.8764 & 4.67  \\
        SSE & 0.02 & 10 & 7233 & 10 & 4.930 & 0.0984 \\
        SeBa & 0.02 & 10 & 7285 & 10 & 4.938 & 0.133 \\
        \hline
        \noalign{\smallskip}
        \noalign{\smallskip}
    \end{tabular}
\end{table}

\section{Cluster age}\label{sec:isochrones}
The method used to calculate the age of the Hyades and Pleiades clusters is described in this section. Firstly, a set of $N$ stars following an initial mass function (IMF) is generated, where $N$ is the number of stars that we set equal to the amount of data points in the dataset. The Hyades dataset consists of 1339 measured stars and Hyades has 887 data entries. Here we adopt the empirically derived Salpeter IMF \citep{1955ApJ...121..161S} given by
\begin{eqnarray}
    \zeta(m) \Delta m = \zeta_0 \left(\frac{m}{M\Sun}\right)^{\alpha} \left(\frac{\Delta m}{M\Sun}\right),
\end{eqnarray} where the exponent $\alpha = -2.35$. In AMUSE the SalpeterIMF implementation yields a mass within the range 0.1 - 125 M\Sun. The IMF shows clearly that higher mass stars are more rare, which we expect. Using this stellar population, generated at the same point in time, we start the stellar evolution using the SSE algorithm for a given metallicity. We adopt $z = 0.14$ \citep{1998A&A...331...81P} as Hyades metallicity, and for the Pleiades we assume $z = 0.02$, which is supported by \citet{2009AJ....138.1292S} within the error estimation.

    Next, we obtained observed color-magnitude diagrams of the Hyades and Pleiades cluster from the online version of the database for galactic open clusters, BDA \citep{1995ASSL..203..127M}, accessible at \citep{webda}. These photometric datasets contain observed apparent magnitudes in the V band and values of B-V. A measure of the `reddening' of 0.010 magnitude is provided by WEBDA for the Hyades, but no measure of the reddening is available for the Pleiades. This reddening might be a measure of the interstellar extinction but, since we were unsure and we only have a value for one cluster, we did not include this value in the analysis. 

    To allow for fitting of the numerically calculated HR diagrams, the color-magnitude measures have to be converted to luminosity and $T_{\rm eff}$. Alternatively, the numerical luminosity may be converted to apparent visual magnitude in the V-band and the effective temperature to a corresponding B-V value. We chose the former using equations~\eqref{eq:lum}~and~\eqref{eq:teff}. For the luminosity $L$ we find


\begin{eqnarray}
    \log_{10} \left(\frac{L_{\rm star}}{L\Sun}\right) = \frac{(M_{V,\rm star}-M_{V,\odot})}{-2.5} - \log_{10} \left( \left(\frac{d_{\rm star}}{d\Sun}\right)^2\right) \label{eq:lum},
\end{eqnarray}
where $M$ the observed apparent magnitude in the V-band and $d$ the distance. Here we adopt $M_{V, \odot} = -26.73$ \citep{1957ApJ...126..266S} and $d_{\rm star} = 46.34 \text{pc}$ for the Hyades \citep{1998A&A...331...81P} and $d_{\rm star} = 120.2 \text{pc}$ for the Pleiades \citep{2009A&A...497..209V}. To obtain the temperature from B and V photometric filter measurements, we use the equation derived by \citet{2012EL.....9734008B}
\begin{eqnarray}
    T = 4600 \left(\frac{1}{0.92 (B-V) + 1.7} + \frac{1}{0.92(B-V) + 0.62} \right) \label{eq:teff}.
\end{eqnarray}

Fig.~\ref{fig:bestfit} shows the converted observations in red and the numerical values in green. In this particular plot we show the best fit. To determine which cluster lifetime fits best to the observations we first need to calculate to which simulated point of the HR diagram in the main sequence, giant branch or white dwarf the observed data point is the nearest. We do so using a nearest neighbour algorithm. Next, we calculate two $\chi^2$ values: one for the luminosity and one for the effective temperature. Since the order of magnitude of $\chi^2$ may differ for the luminosity and the effective temperature, we normalize these values such that they are comparable. We calculate the standard deviation of all temperatures in the dataset first by
\begin{eqnarray}
    \sigma = \sqrt{\frac{\sum_{t=0}^{\text{starcount}} (T_t - T_{\rm average})^2}{\text{starcount}}}, \label{eq:stdev} 
\end{eqnarray}where $T_t$ is the temperature of star a star at time $t$, $T_{\rm average}$ is the temperature averaged over all times, and `starcount' is the total number of stars in the dataset. Using the standard deviation we can now obtain the normalized $\chi^2$, using
\begin{eqnarray}
    \chi^2_{\rm normalized} = \left( \frac{T_{\rm simulated} - T_{\rm observed}}{\sigma} \right)^2.
\end{eqnarray}
This same algorithm is used for the luminosities but with the $T$ replaced by $L$. Finally, the normalized $\chi^2$ values are added assuming both temperature and luminosity contribute equally (i.e. have the same weight) to the fit.

Using the method described in the previous paragraph we have obtained Fig.~\ref{fig:chisq} showing the chi squared values for the cluster age in a range from 0 Myr to 1000 Myr. Note that initially our calculated $\chi^2$ is very low compared, then quickly rises, drops to a local minimum and then rises again. We contribute this behaviour to the fact from 0 to 200 Myr the heaviest stars are not yet evolved. This means that there are no stars in the giant branch and asymptotic giant branches. The luminosities are up to three decades higher for those stars causing the rise in $\chi^2$. Those stars either end as white dwarfs or go supernova, to create neutron stars or black holes. Once the white dwarfs start ascending to the lower left of the HR diagram, the main sequence begins to show show a turnoff towards the giant branches, and the giant branch just post Hertzsprung gap starts to be populated, $\chi^2$ decreases again. We expect the ZAMS turnoff point to be the most significant feature to determine the cluster age. To compare, Fig.~\ref{fig:ZAMS} clearly shows the ZAMS for a cluster of 0 Myr without a turnoff towards the giant branches. Moreover, the lower left corner contains no white dwarfs. Another interesting $\chi^2$ value is found at 810 Myr, where a single point breaks the trend. We discard this point for this reason and conclude that the age of the Hyades is 680 Myr. NB, this analysis has only been done once. To calculate a statistical average we would like to conduct the experiment multiple times using several Salpeter IMFs. We can therefore only state that the age of the Hyades according to the literature is 625 $\pm$ 50 Myr \citep{1998A&A...331...81P}. Both values are the same order of magnitude and differ roughly ten percent. However, we cannot say if our findings are in agreement within the error measures as our error measures are unknown.

For the Pleiades the same analysis has been performed. Fig.~\ref{fig:chisq_pleiades} shows the normalized chi squared for this simulation. The most likely cluster age is 100 Myr ($\chi^2 = 7.312$) or 230 Myr ($\chi^2 = 12.74$). For higher cluster ages $\chi^2$ increases significantly. We therefore conclude that the most likely cluster age is 100 Myr. The literature cluster age of the Pleiades is 75 - 150 Myr \citep[e.g.][]{1998ApJ...497..253U}.

\begin{figure}
    \centering
    \includegraphics[width=\hsize]{img/test680.png}
    \caption{HR diagram of the Hyades cluster at 680 Myr. This simulated cluster age yields the best fit to the observed data (i.e. $\chi^2$ is at a local minimum). For the simulation (green dots) we have assumed 1339 stars of metallicity $z=0.14$ following a Salpeter IMF. The red dots indicate the observed color-magnitude diagram converted to luminosity and effective temperature.}\label{fig:bestfit}
\end{figure}

\begin{figure}
    \centering
    \includegraphics[width=\hsize]{img/fitness_over_time.png}
    \caption{Plot of $\chi^2$ for each simulated cluster age of the Hyades cluster. To obtain this value we summed the least squares values of the luminosity and temperature after normalization. Note that $\chi^2$ begins at a low value because all stars are ZAMS, thus, there are no stars in the giant branches. The latter category have luminosities up to three decades higher than the current vertical axis of this figure, causing $\chi^2$ to increase once stars evolve off the main sequence into the giant branches. The local minimum observed at 680 Myr is the cluster age. Also note the lower $\chi^2$ value at 810 Myr. This point clearly breaks the trend and for that reason we concluded that this is not the cluster age.}\label{fig:chisq}
\end{figure}


\begin{figure}
    \centering
    \includegraphics[width=\hsize]{img/fitness_over_time_pleiades.png}
    \caption{Plot of $\chi^2$ for each simulated cluster age of the Pleiades cluster. To obtain this value we summed the least squares values of the luminosity and temperature after normalization. }\label{fig:chisq_pleiades}
\end{figure}



\begin{figure}
    \centering
    \includegraphics[width=\hsize]{img/test0.png}
    \caption{HR diagram of the Hyades cluster at 0 Myr. Note that since this cluster has just been generated it consists of ZAMS only. Note that main sequence follows a straight line in comparison to the main sequence cut off towards the giant branches for older clusters. This is the primary indicator to determine the cluster age, together with the white dwarfs in the lower left region and the red giants just after the Hertzsprung Gap.}\label{fig:ZAMS}
\end{figure}


\section{Cluster mass-loss} \label{sec:massloss}
% \begin{figure}
%     \centering
%     \includegraphics[width=\hsize]{img/mass_over_time.png}
%     \caption{REPLACE}\label{fig:mass_over_time
% \end{figure}
As stars age they loose mass. By simulating a group of stars with an appropriate distribution of different masses for the stars we can study how a cluster of stars changes in mass. The changing mass can have further repercussion for the gravitational interaction within a cluster (see section~\ref{sec:SE_GD}). 

We have initialized a set of stars with the Salpeter IMF. In Fig.~\ref{fig:massfraction} we can see that over a 1000 Myr simulation the cluster loses about 16\% of its mass. It should be noted that we are strictly talking about mass inside stars; clusters can have mass that is contained in things like dust and gas clouds. Specifically at 680 Myr (the age derived in section~\ref{sec:isochrones}) the fraction of mass retained in stars relative to the starting mass was 84\%. 

This simulation was done with parameters simulating the Hyades. The present day total mass of the Hyades cluster is about 400 M\Sun \citep{2009AIPC.1094..497B}. This would imply an initial cluster mass of $400 M\Sun/0.84 \approx 476 M\Sun$ if we want to end up with a mass of 400 M\Sun \, at 680 Myr. 
\begin{figure}
    \centering
    \includegraphics[width=\hsize]{img/massfraction_over_time.png}
    \caption{Fraction of initial mass present in stars of a cluster over time.}\label{fig:massfraction}
\end{figure}

\section{Stellar Evolution \& Gravitational Dynamics} \label{sec:SE_GD}
Clusters are groups of stars that to some extent are gravitationally bound to each other. Globular clusters can be quite tightly bound while open cluster can be bound somewhat more loosely. By simulating a cluster's gravitational interaction over time and finding how many stars remain bound we can study how strongly the stars are bound to the rest of the cluster and how this changes at different ages of the cluster. By using N-body codes the behavior of gravitationally bound and unbound stars can be studied computationally.

Just running an N-body code on a simulated cluster makes the implicit assumption that the stars are static objects in everything but position and velocity. In particular, it makes the assumption that the masses of the stars do not change. In reality stars change in mass as they progress through their life, which may have a significant impact on the way they gravitationally interact with the rest of a cluster.

We can integrate stellar evolution code with an N-body simulation to make our simulation more accurate. By simulating each star's evolution over time we can use this to determine the mass of each star over time. These masses can in turn be used as the input of an N-body code. By alternatively simulating stellar evolution and gravitational interaction the N-body code will now use different masses for the same star over time.

We have run the simulation with the Barnes-Hut Tree N-Body simulation code \citep{1986Natur.324..446B} for a total period of 1000 Myr. The stellar evolution code being used is SSE. The stellar evolution is run at 10 Myr increments. The N-Body code then takes the new masses and simulates the gravitational dynamics for 10 Myr as well, with a time step of 0.1 Myr (i.e. it uses 100 time steps to move the gravitational simulation forward by 10 Myr).

To find the stars that are considered gravitationally bound the HOP algorithm is used \citep{1998ApJ...498..137E}. Based on this algorithm the number of bound stars in the cluster can be seen in Fig.~\ref{fig:bound_stars}. The cluster only looses about 7 stars at the start of the simulation and stabilizes at 7-9 unbounded stars and over 1300 that remain bounded. 

Because the bound stars metric doesn't show a lot of change instead the clusters radius over time might be a more interesting metric. The most naive way to measure radius is to take the maximum distance a star has to the centre of the cluster. As can be seen in Fig.~\ref{fig:max_radius} this radius increases roughly linearly over time. Because it is determined by the outermost star, this would be consistent with a system where there is at least one star that is relatively unbound by the clusters gravity and is moving away from its centre at a roughly constant speed. 

More interesting is the virial radius. For our simulated cluster this was at 10 light years (about 3 parsecs), matching the core radius of the Hyades cluster \citep{2009AIPC.1094..497B}. As can be seen in Fig.~\ref{fig:virial_radius} it has a large jump at the star of the simulation and then a more slowly increasing trend.

Finally we can look at what phases of their lives the stars in the simulation are over time. We group the stars into three categories: main sequence stars, giant stars and remnant stars. In Fig.~\ref{fig:phases} it can be seen that the fast majority of the stars are on the main sequence and remain their during the simulation. When we only look at giant and remnant stars in Fig.~\ref{fig:phases_without_ms} we can see that while the number of remnant stars steadily increases, the number of giant stars at any single point in time remains relatively constant. This implies that the rate of stars entering their giant phase and burning out from the giant phase is roughly equal.
\begin{figure}
    \centering
    \includegraphics[width=\hsize]{img/cluster_max_radius.png}
    \caption{Radius of the cluster, determined by star furthest away from centre.}\label{fig:max_radius}
\end{figure}

\begin{figure}
    \centering
    \includegraphics[width=\hsize]{img/cluster_virial_radius.png}
    \caption{Virial radius of cluster over time.}\label{fig:virial_radius}
\end{figure}

\begin{figure}
    \centering
    \includegraphics[width=\hsize]{img/bound_stars.png}
    \caption{Number of stars gravitationally bound by cluster.}\label{fig:bound_stars}
\end{figure}

\begin{figure}
    \centering
    \includegraphics[width=\hsize]{img/stellar_phases_counts.png}
    \caption{Stars in three stellar evolution categories over time.}\label{fig:phases}
\end{figure}

\begin{figure}
    \centering
    \includegraphics[width=\hsize]{img/stellar_phases_counts_without_main_sequence.png}
    \caption{Stars in two stellar evolution categories over time (excluding main sequence stars).}\label{fig:phases_without_ms}
\end{figure}

\section{Discussion}\label{sec:discussion}
We must assume that all stars in the cluster are formed at the same point in time in order to use this method. This assumption might not be true as observations show that star formation in clusters might be at different times, causing multiple populations in stellar clusters \citep{2009IAUS..258..233P}.

Binaries are not included in our analysis, although research has shown that almost one in three stars in the Milky Way is in a binary \citep{2006ApJ...640L..63L}. This is a significant fraction that should be included in the analysis.

We have assumed the stars to follow the Salpeter IMF. Other, more recent IMFs are available \citep[e.g.][]{1979ApJS...41..513M, 2001ASPC..228..187K, 2001MNRAS.322..231K} and may be taken into the analysis as a variable.
For instance, \citet{2001MNRAS.322..231K} is concerned that a universal initial mass function (IMF) is not intuitive yet evidence lacks to support of a variable IMF. Moreover, different slopes are introduced for stars below 0.08 M\Sun and between 0.08 M\Sun and 0.5 M\Sun. In our analysis, we have produced stars with masses below 0.5 M\Sun so this could influence the results. In fact, the mean mass of the Salpeter IMF is 0.35 M\Sun, which is in the 0.08 - 0.5 M\Sun mass range requiring an alternate power law index $\alpha$. This influence should be studied in depth but lacks in our analysis.

We would like to emphasize that the current value of the Solar metallicity is highly debated at present. AMUSE by defaults chooses $z = 0.02$ as Solar metallicity. This value lies within the range of possible Solar metallicities of $z= 0.0187$ to $z = 0.0239$ \citep{2007ApJ...670..872C}. Note, however, that three-dimensional hydrodynamical simulations of the Solar atmosphere yield a significantly lower solar metallicity of $z = 0.0122$ \citep{2006CoAst.147...76A}. At present, most stellar evolution models assume a one dimensional (spherically symmetric) geometry. This neglects three dimensional convective flows that might have a significant influence on the stellar structure. We did not perform a detailed analysis of different stellar metallicities and we merely state that the reader should be cautious regarding the Solar metallicity and note that we adopt $z = 0.02$ as the Solar metallicity because AMUSE defaults to said value. This discussion is highly relevant because, for example, metallicity has a profound influence on the stellar evolution \citep[e.g.][]{1960ApJ...131..598S}. Having said so, the metallicity for the Hyades has been obtained from the literature. We therefore state that for isochrone fitting of the Hyades the metallicity is not a point of discussion. Likewise, the Pleiades metallicity was derived from measured values in the literature.

We have noticed that the dataset has double entries. This occurs, for instance, when multiple publications governing the same star exist in the literature. Consequently, some stars have erroneously been given a higher weighing factor in the least squares determination.  We, however, neglect this because (e.g. in the Hyades dataset) we have found that most stars have double entries and stars with double entries span the entire spectrum. We therefore assume that this does not significantly influence the end-result. On the other hand, future analysis can be improved by averaging the data points and calculating the observational error to be used in the least squares averaging. This will result in improving the weight factor as observations contradicting each other will be given a lower weight factor whereas observations in agreement are given a higher weight factor due to the $chi^2$ calculations.

\section{Conclusions}\label{sec:conclusions}
\begin{enumerate}
    \item We find that the age of the Hyades is 680 Myr, as compared to 625 $\pm$ 50 Myr obtained from the literature.
    \item At present, the Hyades has lost 16\% of its mass due to stellar evolution.
    \item Within AMUSE it is fairly easy to set up a bridge between stellar evolution code and gravitational dynamics. In the future mass loss could be added as gas in the galaxy/cluster and taken into account using the hydrodynamics code available in AMUSE.
    \item As a simulated cluster ages, it accumulates remnant stars, but will have a roughly constant number of stars in a giant phase at any point in time.
   \end{enumerate}


\begin{acknowledgements}
The authors are grateful for the help of their supervisors Edwin van der Helm, MSc and Prof. dr. S.F. Portegies Zwart. \\

This research has made use of the WEBDA database, operated at the Department of Theoretical Physics and Astrophysics of the Masaryk University. \\

This research has made use of NASA's Astrophysics Data System.
\end{acknowledgements}


%-------------------------------------------------------------------

\bibliographystyle{aa}
%\setlength{\bibsep}{0pt} % Remove whitespace in bibliography.
\bibliography{CA_SE_TLRH_s1603221_SS_s1617451_report_aa}
\end{document}
