\documentclass[a4paper]{article}
\usepackage{amsmath}
\usepackage{amssymb}
\usepackage{graphicx}
\usepackage{a4wide}
\usepackage{url}
\usepackage{pdfpages}

\usepackage{natbib}
\citestyle{aa}
\bibpunct{(}{)}{;}{a}{}{,}

\newcommand{\mytilde}{\raise.17ex\hbox{$\scriptstyle\mathtt{\sim}$}}
\setlength{\parindent}{0pt}

% Listings to import code into LaTeX
\usepackage{listings}
\usepackage{color}
\usepackage{textcomp}
\definecolor{listinggray}{gray}{0.9}
\definecolor{lbcolor}{rgb}{0.9,0.9,0.9}
\lstset{
        backgroundcolor=\color{lbcolor},
        tabsize=4,
        rulecolor=,
        language=bash,
        basicstyle=\scriptsize,
        upquote=true,
        aboveskip={1.5\baselineskip},
        columns=fixed,
        showstringspaces=false,
        extendedchars=true,
        breaklines=true,
        prebreak = \raisebox{0ex}[0ex][0ex]{\ensuremath{\hookleftarrow}},
        frame=single,
        showtabs=false,
        showspaces=false,
        showstringspaces=false,
        identifierstyle=\ttfamily,
        keywordstyle=\color[rgb]{0,0,1},
        commentstyle=\color[rgb]{0.133,0.545,0.133},
        stringstyle=\color[rgb]{0.627,0.126,0.941},
}
% End of Listings

\begin{document}
\begin{figure}[h!]
\begin{center}
\includegraphics{UvA_Logo_Image_EN.jpg} \\
\includegraphics{UvA_Logo_Text_EN.jpg} \\
\includegraphics{API_Logo_Text.pdf}
\end{center}
\end{figure}

\begin{center}
\line(1,0){420} \\
\huge \textbf{Computational Astrophysics (CA) \\
Simulating Ultra Compact Dwarf Galaxies with AMUSE} \\
\line(1,0){420}
\end{center}

\vfill

%\begin{figure}[h!]
%\begin{center}
%\includegraphics[width=12cm]{graphgraphMyGraph}
%\end{center}
%\end{figure}
%\addtocounter{figure}{-1} % start counting at figure 1 again


\begin{table}[h]
\begin{center}
\begin{tabular}{lp{5cm}l}
\textit{Author:} & & \emph{Supervisor:} \\
Timo Halbesma, 1603221 & Shabaz Sultan, 1617451 & Prof.dr. S.F. Portegies Zwart\\
& & Edwin van der Helm, MSc\\
\end{tabular}
\end{center}
\end{table}

% End of titlepage

\newpage

%\lstinputlisting[caption={Solution for CA / Gravitational
%Dynamics\label{list:main}}]{../GravitationalDynamics/CA_GD_TLRH_s1603221.py}

%\lstinputlisting[caption={Solution for CA / Gravitational
%Dynamics\label{list:plotter}}]{../GravitationalDynamics/plot_nbody.py}

%\lstinputlisting[caption={Solution for CA / Gravitational
%Dynamics\label{list:solver}}]{../GravitationalDynamics/solve_nbody.py}

\section*{Assignment 1A}
In Figure~\ref{fig:runtime_BH} and Figure~\ref{fig:dE_BH} respectively
Figure~\ref{fig:runtime_Herm} and Figure~\ref{fig:dE_Herm} the required plots for this
assignment can be found for the Barnes Hut Tree \citep{1986Natur.324..446B} and the 4th order
Hermite Predictor/Corrector algorithm \citep{1995ApJ...443L..93H}. The assignment sheet was
ambiguous wether to use BHTree or Hermite as default for this assignment, so we
considered both.\\


\textbf{BHTree} \\
The cluster size $r$ is passed to the nbody\_integrator function as parameter
named `rcl'. This parameter is only used in amuse.units.nbbody\_system, which is
responsible for the creation of bodies within a convined cluster with half-mass
radius `rcl'. The distance between individual particles could increase  as the
cluster half mass radius increases. This maximum distance between particles
scales linearly with the cluster half mass radius. The distance between
particles can be found in the force, but the number of times the force is
calculated does not depend on it. If the cluster size increases and the number
of paricles is unchanged, then at a certain point in time more particles could
be further away. In that case more particles will be bundles together in the
same BHTree, thus, in principle the calculation time could decrease as the
cluster size r increases if particles move further away. 

\begin{figure}[h!]
\begin{center}
\includegraphics[height=12cm]{../GravitationalDynamics/CA_GD_TLRH_s1603221_SS_s1617451_BHTree_runtime.png}
\caption{Wall-clock time as a function of both N and integration end time.}
\label{fig:runtime_BH}
\end{center}
\end{figure}

\begin{figure}[h!]
\begin{center}
\includegraphics[height=12cm]{../GravitationalDynamics/CA_GD_TLRH_s1603221_SS_s1617451_BHTree_dE.png}
\caption{Relative energy error as a function of both N and integration end time.}
\label{fig:dE_BH}
\end{center}
\end{figure}

\textbf{Hermite} \\
Dependance of wall-clock time on cluster size REPLACEREPLACE
\begin{figure}[h!]
\begin{center}
\includegraphics[height=12cm]{../GravitationalDynamics/CA_GD_TLRH_s1603221_SS_s1617451_Hermite_runtime.png}
\caption{Wall-clock time as a function of both N and integration end time.}
\label{fig:runtime_Herm}
\end{center}
\end{figure}

\begin{figure}[h!]
\begin{center}
\includegraphics[height=12cm]{../GravitationalDynamics/CA_GD_TLRH_s1603221_SS_s1617451_Herimte_dE.png}
\caption{Relative energy error as a function of both N and integration end time.}
\label{fig:dE_Herm}
\end{center}
\end{figure}

\section*{Assignment 1B}

\section*{Assignment 1C}
HUAYNO \citep{2012NewA...17..711P}

\section*{Assignment 1D}
HOP Bound Mass \citep{1998ApJ...498..137E}

\section*{Assignment 1E}

\section*{Assignment 1F}

\newpage
\bibliographystyle{apj} 
%\setlength{\bibsep}{0pt} % Remove whitespace between BibTeX-entries
\bibliography{apj-jour,CA_GD_TLRH_s1603221_SS_s1617451_report}
% \citep{label01}, verwijzing stijl: (Naam 0000) (naam en jaartal).
% \citet{label01} verwijzing stijl Naam (0000), naam en (jaartal).


\end{document}
